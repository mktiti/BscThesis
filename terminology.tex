%\documentclass{book}
%\usepackage[utf8]{inputenc}

%\begin{document}

	\chapter*{Glossary}

	\paragraph{Game}
	\begin{itemize}
		\item A playable entity for which playing bots can be written. It consists of a game API, a game engine, and other meta data including a unique name and player number information. 
		
		\item An actual event of a game being played by a bot or bots and guided by the game's engine.
	\end{itemize}
	
	\paragraph{Challenge} A single player game, whose result (if not an error) is a whole number representing the score received by the playing bot, and an optional maximum points limit.
  
  	\paragraph{Match} A two player, competitive game, which can be played by two different bots, and results in a three way outcome or an error.
  
  	\paragraph{Game Runtime API} A java library containing type definitions that allows games and bots to be created. It is to be used as a provided (non included) dependency for the game engine, game api, and bot implementations.
  	
  	\paragraph{Engine Runtime API} A java library containing type definitions and abstract base implementations that allow game engines to be written. It is to be used as a provided (non included) dependency for the game engine.
  
  	\paragraph{(Game) Engine} An executable library that provides the necessary logic and implementation for a game to work according to its rules.
  
  	\paragraph{Game API} A java library that contains type definitions and utility code that enable users to create bots for the specific game.
  
  	\paragraph{Bot interface} An interface that extends the \code{BotInterface} marker interface defined in the game runtime api. It defines the game specific methods, through which the game engine can interact with the bot. 
  	
  	\paragraph{Bot} A virtual player built to play a specific game --- alone if that is a challenge or against another bot if that is a match. It is implemented as a class realizing the game's bot interface.
    	
  	\paragraph{Actor} A collective term referring to a game engine and the bot or bots playing its game in a given challenge or match.
  
  	\paragraph{Actor client} A separate process which handles and interacts with a single actor throughout a game.
  
%\end{document}